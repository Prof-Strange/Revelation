% ----------------------------------------------------------------
% AMS-LaTeX Paper ************************************************
% **** -----------------------------------------------------------
\documentclass[10pt,reqno]{amsart}
\usepackage{bbm}
\usepackage{mathrsfs}
\usepackage{amsfonts} %%% i.e. use 12pt type
\usepackage[dvipsnames,usenames]{color}
\textwidth=13.5cm %%% in the preamble; this will require
\baselineskip=17pt %%% after \begin{document}
\usepackage{graphicx,latexsym,bm,amsmath,amssymb,verbatim,multicol,lscape}
\usepackage{enumerate}
% ----------------------------------------------------------------
\vfuzz2pt % Don't report over-full v-boxes if over-edge is small
\hfuzz2pt % Don't report over-full h-boxes if over-edge is small
% THEOREMS -------------------------------------------------------
\newtheorem{thm}{Theorem} [section]
\newtheorem{lem}{Lemma}[section]
\newtheorem{pro}{Proposition}[section]
\newtheorem{cor}{Corollary}[section]
\newtheorem{exm}{Example}[section]
\theoremstyle{definition}
\newtheorem{defn}{Definition}[section]
\theoremstyle{remark}
\newtheorem{rem}{Remark}[section]
\newtheorem{con}[thm]{Conjecture}
\numberwithin{equation}{section}
\DeclareMathOperator{\Tr}{Tr}
\DeclareMathOperator{\ind}{ind}
\allowdisplaybreaks
% MATH -----------------------------------------------------------
\newcommand{\norm}[1]{\left\Vert#1\right\Vert}
\newcommand{\abs}[1]{\left\vert#1\right\vert}
\newcommand{\set}[1]{\left\{#1\right\}}
\newcommand{\Real}{\mathbb R}
\newcommand{\eps}{\varepsilon}
\newcommand{\To}{\longrightarrow}
\newcommand{\BX}{\mathbf{B}(X)}
\newcommand{\A}{\mathcal{A}}
\newcommand{\SSS}{\stackrel}
% ----------------------------------------------------------------
\begin{document}

\begin{defn}
Let $A$ be an object set.We define a total order on $A$ by the one-to-one mapping
$$\tau:A\rightarrow \{1,2,...,|A|\}$$
that is,$\forall x\not=y\in A,x<y\Leftrightarrow \tau(x)<\tau(y)$.

Let $L$ be the set that contains all of the $\tau$ .
\end{defn}

\begin{defn}
A function $f:L^n\rightarrow A$ is called a choice function.
\end{defn}

\begin{defn}
A choice function $f$ can be $strategically\quad manipulated$ by voter $i$ if for some $\tau_1,...,\tau_n\in L$ and some $\tau'_i\in L$ we have that $\tau_i(a)<\tau_i(a')$ where $a=f(\tau_1,...,\tau_n)$ and $a'=f(\tau_1,...\tau'_i,...,\tau_n)$ . That is, voter $i$ that prefers $a'$ to $a$ can ensure that $a'$ gets chosen rather than $a$ by strategically misrepresenting his preferences to be $\tau'_i$ rather than $\tau_i$ . $f$ is called $incentive\quad compatible$ if it cannot be manipulated.
\end{defn}

\begin{defn}
A choice function $f$ is monotone if $f(\tau_1,...,\tau_i,...,\tau_n)=a\not=a'=f(\tau_1,...,\tau'_i,...,\tau_n)$ implies($\Rightarrow$) that $\tau_i(a')<\tau_i(a)$ and $\tau'_i(a)<\tau'_i(a')$ . That is if the choice changed from $a$ to $a'$ when a single voter $i$ changed his vote from $\tau_i$ to $\tau'_i$ then it must be because he switched his preference between $a$ and $a'$ .
\end{defn}


\begin{defn}
Let $\vec{v}=(v_1,...,v_n)$ be an $n$-dimentional vector. We will denote the $(n-1)$-dimentional vector in which the $i$-th coordinate is removed by $\vec{v_{-i}}=(v_1,...,v_{i-1},v_{i+1},...,v_n)$ . Thus we have three equivalent notations:$\vec{v}=(v_1,...,v_n)=(v_i,\vec{v_{-i}})$ .
\end{defn}

\begin{defn}
A game with(independent private values and) strict incomplete information for a set of $n$ players is given by the following ingredients:

(i)For every player $i$ , a set of $actions$ $X_i$.

(ii)For every player $i$ , a set of $types$ $T_i$ . A value $t_i\in T_i$ is the private information that $i$ has.

(iii)For every player $i$ , a $utility$ $function$ $u_i:T_i\times X_1\times ...\times X_n\rightarrow \mathbb{R}$ , where $u_i(t_i,x_1,...,x_n)$ is the utility achieved by player $i$ , if his type is $t_i$ , and the profile of actions taken by all players is $x_1,...,x_n$ . 
\end{defn}

\begin{defn}
(i)A strategy of a player $i$ is a function $s_i:T_i\rightarrow X_i$ .

(ii)A strategy $s_i$ is a (weakly) dominant strategy if for every $t_i$ we have that the action $s_i(t_i)$ is a dominant strategy in the full information game defined by $t_i$ . Formally:For all $t_i$, all$x_{-i}$ and all $x_i'$ we have that $u_i(t_i,s_i(t_i),x_{-i})\ge u_i(t_i,x_i',x_{-i})$. A profile $s_1,...,s_n$ is called a dominant strategy equilibrium if each $s_i$ is a dominant strategy.
\end{defn}

\begin{defn}
(i)A mechanism for $n$ players is given by

(a)players' type spaces $T_1,...,T_n$ ,

(b)players' action spaces $X_1,...,X_n$ ,

(c)an alternative set $A$ ,

(d)an outcome function $a:X_1\times...\times X_n\rightarrow A$ and ,


The game with strict incomplete information induced by the mechanism is given by using the types spaces $T_i$ , the action spaces $X_i$ , and the utilities $u_i(t_i,x_1,...,x_n)$ .

(ii)The mechanism implements a choice function $f:T_1\times...\times T_n\rightarrow A$ in dominant strategies if for some dominant strategy equilibrium $s_1,...,s_n$ of the induced game , where $s_i:T_i\rightarrow X_i$ , we have that for all $t_1,...,t_n$ , $f(t_1,...,t_n)=a(s_1(t_1),...,s_n(t_n))$ .

\end{defn}

\begin{pro}[Revelation Principle]
If there exists an arbitrary mechanism that implements f in dominant strategies , then there exists an incentive compatible mechanism that implements f. The payments of the players in the incentive compatible mechanism are identical to those , obtained at equilibrium , of the original mechanism.
\end{pro}

\begin{proof}
The new mechanism will simply simulate the equilibrium strategies of the players . That is , let $s_1,...,s_n$ be a dominant strategy equilibrium of the original mechanism , we define a new direct revelation mechanism : $f(t_1,...,t_n)=a(s_1(t_1),...,s_n(t_n))$ . Now since each $s_i$ is a dominant strategy for player $i$ , then for every $t_i,x_{-i},x_i'$ we have that $u_i(t_i,a(s_i(t_i),x_{-i}))\ge u_i(t_i,a(x_i',x_{-i}))$ . Thus in particular this is true for all $x_{-i}=s_{-i}(t_{-i})$ and any $x_i'=s_i(t_i')$ , which gives the definition of incentive compatibility of the mechanism $(f,u_1',...,u_n')$ .

\end{proof}
\end{document}