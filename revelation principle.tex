% ----------------------------------------------------------------
% AMS-LaTeX Paper ************************************************
% **** -----------------------------------------------------------
\documentclass[10pt,reqno]{amsart}
\usepackage{bbm}
\usepackage{mathrsfs}
\usepackage{amsfonts} %%% i.e. use 12pt type
\usepackage[dvipsnames,usenames]{color}
\textwidth=13.5cm %%% in the preamble; this will require
\baselineskip=17pt %%% after \begin{document}
\usepackage{graphicx,latexsym,bm,amsmath,amssymb,verbatim,multicol,lscape}
\usepackage{enumerate}
% ----------------------------------------------------------------
\vfuzz2pt % Don't report over-full v-boxes if over-edge is small
\hfuzz2pt % Don't report over-full h-boxes if over-edge is small
% THEOREMS -------------------------------------------------------
\newtheorem{thm}{Theorem} [section]
\newtheorem{lem}{Lemma}[section]
\newtheorem{pro}{Proposition}[section]
\newtheorem{cor}{Corollary}[section]
\newtheorem{exm}{Example}[section]
\theoremstyle{definition}
\newtheorem{defn}{Definition}[section]
\theoremstyle{remark}
\newtheorem{rem}{Remark}[section]
\newtheorem{con}[thm]{Conjecture}
\numberwithin{equation}{section}
\DeclareMathOperator{\Tr}{Tr}
\DeclareMathOperator{\ind}{ind}
\allowdisplaybreaks
% MATH -----------------------------------------------------------
\newcommand{\norm}[1]{\left\Vert#1\right\Vert}
\newcommand{\abs}[1]{\left\vert#1\right\vert}
\newcommand{\set}[1]{\left\{#1\right\}}
\newcommand{\Real}{\mathbb R}
\newcommand{\eps}{\varepsilon}
\newcommand{\To}{\longrightarrow}
\newcommand{\BX}{\mathbf{B}(X)}
\newcommand{\A}{\mathcal{A}}
\newcommand{\SSS}{\stackrel}
% ----------------------------------------------------------------
\begin{document}



\begin{defn}
Let $\vec{v}=(v_1,...,v_n)$ be an $n$-dimentional vector. We will denote the $(n-1)$-dimentional vector in which the $i$-th coordinate is removed by $\vec{v_{-i}}=(v_1,...,v_{i-1},v_{i+1},...,v_n)$ . Thus we have three equivalent notations:$\vec{v}=(v_1,...,v_n)=(v_i,\vec{v_{-i}})$ .
\end{defn}


\begin{defn}
A (direct revelation) mechanism is a choice function $f$ and a vector of utility functions $u_1^f,...,u_n^f$
\begin{eqnarray}
f:V_1\times ... \times V_n \rightarrow A\\
u_i^f :V_1\times ... \times V_n \rightarrow \mathbb{R}
\end{eqnarray}
\end{defn}


\begin{defn}
A mechanism $(f,u_1^f,...,u_n^f) $ is called incentive compatible if $\forall i, \exists  v_i \in V_i, \forall v_i' \in V_i,  $  % the second quantifier should not be \froall but be \exists
\begin{equation}
u_i^f(v_i,v_{-i} )\geq u_i^f(v'_i,v_{-i} )
\end{equation}
\end{defn}

\begin{defn}
A game with(independent private values and) strict incomplete information for a set of $n$ players is given by the following ingredients:

(i)For every player $i$ , a set of $actions$ $X_i$.

(ii)For every player $i$ , a set of $types$ $T_i$ . A value $t_i\in T_i$ is the private information that $i$ has.

(iii)For every player $i$ , a $utility$ $function$ $u_i: T_i \times X_1\times ...\times X_n\rightarrow \mathbb{R}$ , where $u_i(t_i,x_1,...,x_n)$ is the utility achieved by player $i$ , if his type is $t_i$, and the profile of actions taken by all players is $x_1,...,x_n$ . 
\end{defn}

\begin{defn}
(i)A strategy of a player $i$ is a function $s_i:T_i\rightarrow X_i$ .

(ii)A strategy $s_i$ is a (weakly) dominant strategy if for every $t_i$ we have that the action $s_i(t_i)$ is a dominant strategy in the full information game defined by $t_i$ . Formally:For all $t_i$, all$x_{-i}$ and all $x_i'$ we have that
\begin{equation}
 u_i(t_i,  s_i(t_i),x_{-i})\ge u_i(t_i, x_i',x_{-i})
\end{equation}
 A profile $s_1,...,s_n$ is called a dominant strategy equilibrium if each $s_i$ is a dominant strategy.
\end{defn}

\begin{defn}
(i)A mechanism for $n$ players is given by

(a)players' type spaces $T_1,...,T_n$ ,

(b)players' action spaces $X_1,...,X_n$ ,

(c)an alternative set $A$ ,

(d)an outcome function $a:X_1\times...\times X_n\rightarrow A$ and ,


The game with strict incomplete information induced by the mechanism is given by using the types spaces $T_i$ , the action spaces $X_i$ , and the utilities $u_i(t_i,x_1,...,x_n)$ .

(ii)The mechanism implements a choice function $f:T_1\times...\times T_n\rightarrow A$ in dominant strategies if for some dominant strategy equilibrium $s_1,...,s_n$ of the induced game , where $s_i:T_i\rightarrow X_i$ , we have that for all $t_1,...,t_n$ , $f(t_1,...,t_n)=a(s_1(t_1),...,s_n(t_n))$ .

\end{defn}

\begin{pro}[Revelation Principle]
If there exists an arbitrary mechanism that implements f in dominant strategies , then there exists an incentive compatible mechanism that implements f. The payments of the players in the incentive compatible mechanism are identical to those , obtained at equilibrium , of the original mechanism.
\end{pro}

\begin{proof}
The new mechanism will simply simulate the equilibrium strategies of the players . That is , let $s_1,...,s_n$ be a dominant strategy equilibrium of the original mechanism , we define a new direct revelation mechanism : 
\begin{eqnarray}
f(t_1,...,t_n):=a(s_1(t_1),...,s_n(t_n)) \\
 u_i^f(t_1,...,t_n):=u_i(t(i), s_1(t_1),...,s_n(t_n)), \text{ where } t(i) := t_i
\end{eqnarray}

 . Now since each $s_i$ is a dominant strategy for player $i$ , then for every $t_i,x_{-i},x_i'$ we have that 
\begin{eqnarray}
 u_i(t(i),   s_i(t_i),x_{-i})\ge u_i(t(i),x_i',x_{-i})\\
 u_i\left(t(i),   s_i(t_i),s_{-i}(t_{-i})  \right) \ge  u_i\left(t(i),   s_i(t'_i),s_{-i}(t_{-i})  \right) \\
 u_i^f(t_i,t_{-i} )\geq u_i^f(t'_i,t_{-i} )
\end{eqnarray}
, which gives the definition of incentive compatibility of the mechanism $(f,u_1^f,...,u_n^f)$ .

\end{proof}
\end{document}